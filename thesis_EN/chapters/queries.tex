\chapter{Queries}
\label{cha:queries}

\subsection{ALTER}
\subsubsection{Description}
Changes read/write throughput on table.\\
If not both but only one of the throughput values should be changed, "0" or "*" must be used.

\subsubsection{Synopsis}
\lstset{language=Java}
\begin{lstlisting}
ALTER TABLE tablename
    SET [INDEX index] THROUGHPUT throughput
ALTER TABLE tablename
    DROP INDEX index [IF EXISTS]
ALTER TABLE tablename
    CREATE GLOBAL [ALL|KEYS|INCLUDE] INDEX global_index [IF NOT EXISTS]
\end{lstlisting}
\begin{itemize}
	\item \textbf{tablename}\\
	name of table
	\item \textbf{throughput}\\
	read/write throughput (read\_throughput, write\_throughput)
	\item \textbf{index}\\
	name of global index
\end{itemize}

\subsubsection{Examples}
\lstset{language=Java}
\begin{lstlisting}
ALTER TABLE foobars SET THROUGHPUT (4, 8);
ALTER TABLE foobars SET THROUGHPUT (7, *);
ALTER TABLE foobars SET INDEX ts-index THROUGHPUT (5, *);
ALTER TABLE foobars SET INDEX ts-index THROUGHPUT (5, *);
ALTER TABLE foobars DROP INDEX ts-index;
ALTER TABLE foobars DROP INDEX ts-index IF EXISTS;
ALTER TABLE foobars CREATE GLOBAL INDEX ('ts-index', ts NUMBER, THROUGHPUT (5, 5));
ALTER TABLE foobars CREATE GLOBAL INDEX ('ts-index', ts NUMBER) IF NOT EXISTS;
\end{lstlisting}

% --------------------------- ANALYZE
\subsection{ANALYZE}
\subsubsection{Description}
Can prefix any query reading or writing data.\\
Will also print out how much capacity was consumed at every part of query.
\subsubsection{Synopsis}
\lstset{language=Java}
\begin{lstlisting}
ANALYZE query
\end{lstlisting}
\vspace{20pt}

\subsubsection{Examples}
\lstset{language=Java}
\begin{lstlisting}
ANALYZE SELECT * FROM foobars WHERE id = 'a';
ANALYZE INSERT INTO foobars (id, name) VALUES (1, 'dsa');
ANALYZE DELETE FROM foobars KEYS IN ('foo', 'bar'), ('baz', 'qux');
\end{lstlisting}
\vspace{40pt}

% --------------------------- CREATE
\subsection{CREATE}
\subsubsection{Description}
Creates new table. Must have: 
\begin{itemize}
	\item exactly one hash key
	\item zero or one range keys (mandatory if having local indexes)
	\item up to five local indexes 
	\item up to five global indexes
\end{itemize}

\subsubsection{Synopsis}
\lstset{language=Java}
\begin{lstlisting}
CREATE TABLE
    [IF NOT EXISTS]
    tablename
    attributes
    [GLOBAL [ALL|KEYS|INCLUDE] INDEX global_index]	
\end{lstlisting}
\begin{itemize}
	\item \textbf{IF NOT EXISTS}\\
	don't through an exception if table already exists
	\item \textbf{tablename} \\
	name of table you want to alter
	\item \textbf{attributes} \\
	attribute declaration like $name$ $data type$  $[key type] $. \\
	Available data types: STRING, NUMBER, BINARY \\
	Available key types : $HASH\_KEY$, $RANGE\_KEY$, $[ALL|KEYS|INCLUDE]$ $INDEX(name) $
	\item \textbf{global\_index}\\
	global index for table
\end{itemize}

\subsubsection{Examples}
\lstset{language=Java}
\begin{lstlisting}
CREATE TABLE foobars (id STRING HASH KEY);
CREATE TABLE IF NOT EXISTS foobars (id STRING HASH KEY);
CREATE TABLE foobars (id STRING HASH KEY, foo BINARY RANGE KEY,
                      THROUGHPUT (1, 1));
CREATE TABLE foobars (id STRING HASH KEY,
                      foo BINARY RANGE KEY,
                      ts NUMBER INDEX('ts-index'),
                      views NUMBER INDEX('views-index'));
CREATE TABLE foobars (id STRING HASH KEY, bar STRING) GLOBAL INDEX
                     ('bar-index', bar, THROUGHPUT (1, 1));
CREATE TABLE foobars (id STRING HASH KEY, baz NUMBER,
                      THROUGHPUT (2, 2))
                      GLOBAL INDEX ('bar-index', bar STRING, baz)
                      GLOBAL INCLUDE INDEX ('baz-index', baz, ['bar'], THROUGHPUT (4, 2));	
\end{lstlisting}
\vspace{40pt}

% --------------------------- DELETE
\subsection{DELETE}
\subsubsection{Description}
Deletes items from table.

\subsubsection{Synopsis}
\lstset{language=Java}
\begin{lstlisting}
DELETE FROM
    tablename
    [ KEYS IN primary_keys ]
    [ WHERE expression ]
    [ USING index ]
    [ THROTTLE throughput ]	
\end{lstlisting}

\begin{itemize}
	\item \textbf{tablename} \\
	name of table
	\item \textbf{primary\_keys} \\
	list of primary keys to the items to be deleted
	\item \textbf{expression} \\
	see SELECT for details about the WHERE clause, Chapter \ref{sssec:select}
	\item \textbf{index} \\
	when WHERE uses an indexed attribute, it allows you to specify an index name for the query
	\item \textbf{THROTTLE} \\
	limit the amount of throughput this query can consume\\
	$(read\_throughput, write\_throughput)$\\
	i.e. $(20\%, 50\%)$ or $(20, 50)$ 
\end{itemize}
\vspace{40pt}

\subsubsection{Examples}
\lstset{language=Java}
\begin{lstlisting}
DELETE FROM foobars; -- This will delete all items in the table!
DELETE FROM foobars WHERE foo != 'bar' AND baz >= 3;
DELETE FROM foobars KEYS IN 'hkey1', 'hkey2' WHERE attribute_exists(foo);
DELETE FROM foobars KEYS IN ('hkey1', 'rkey1'), ('hkey2', 'rkey2');
DELETE FROM foobars WHERE (foo = 'bar' AND baz >= 3) USING baz-index;
	
\end{lstlisting}
\vspace{40pt}


% --------------------------- DROP
\subsection{DROP}
\subsubsection{Description}
Deletes table and all of its items.\\
Should be used carefully!!

\subsubsection{Synopsis}
\lstset{language=Java}
\begin{lstlisting}
DROP TABLE
    [ IF EXISTS ]
    tablename	
\end{lstlisting}

\begin{itemize}
	\item \textbf{IF EXISTS}\\
	don't raise an exception if table doesn't exist
	\item \textbf{tablename} \\
	name of table
\end{itemize}

\subsubsection{Examples}
\lstset{language=Java}
\begin{lstlisting}
DROP TABLE foobars;
DROP TABLE IF EXISTS foobars;	
\end{lstlisting}
\vspace{40pt}

% --------------------------- DUMP
\subsection{DUMP}
\subsubsection{Description}
Prints out the matching CREATE statements for tables.

\subsubsection{Synopsis}
\lstset{language=Java}
\begin{lstlisting}
DUMP SCHEMA [ tablename [, ...] ]
\end{lstlisting}

\begin{itemize}
	\item \textbf{tablename}\\
	name of table(s) whose schema should be dumped.\\
	will dump every table schemas by default if no tablename is given.
\end{itemize}

\subsubsection{Examples}
\lstset{language=Java}
\begin{lstlisting}
DUMP SCHEMA;
DUMP SCHEMA foobars, widgets;	
\end{lstlisting}
\vspace{40pt}

% --------------------------- EXPLAIN
\subsection{EXPLAIN}
\subsubsection{Description}
Meta-query, which is going to print out debug informations.
\subsubsection{Synopsis}
\lstset{language=Java}
\begin{lstlisting}
EXPLAIN query	
\end{lstlisting}
\vspace{20pt}

\subsubsection{Examples}
\lstset{language=Java}
\begin{lstlisting}
EXPLAIN SELECT * FROM foobars WHERE id = 'a';
EXPLAIN INSERT INTO foobars (id, name) VALUES (1, 'dsa');
EXPLAIN DELETE FROM foobars KEYS IN ('foo', 'bar'), ('baz', 'qux');	
\end{lstlisting}
\vspace{40pt}

% --------------------------- INSERT 
\subsection{INSERT}
\subsubsection{Description}
Inserts data into table.

\subsubsection{Synopsis}
\lstset{language=Java}
\begin{lstlisting}
INSERT INTO tablename
    attributes VALUES values
    [ THROTTLE throughput ]
INSERT INTO tablename
    items
    [ THROTTLE throughput ]	
\end{lstlisting}

\begin{itemize}
	\item \textbf{tablename} \\
	name of table
	\item \textbf{attributes} \\
	comma-seperated list of attributes
	\item \textbf{values} \\
	comma-seperated list of data, that should be inserted\\
	must contain same number of items as the attributes param
	$var [, var]...$ 
	\item \textbf{items} \\
	comma-seperated key-value pairs, that should be inserted
	\item \textbf{THROTTLE} \\
	limit the amount of throughput this query can consume\\
	$(read\_throughput, write\_throughput)$\\
	i.e. $(20\%, 50\%)$ or $(20, 50)$
\end{itemize}

\subsubsection{Examples}
\lstset{language=Java}
\begin{lstlisting}
INSERT INTO foobars (id) VALUES (1);
INSERT INTO foobars (id, bar) VALUES (1, 'hi'), (2, 'yo');
INSERT INTO foobars (id='foo', bar=10);
INSERT INTO foobars (id='foo'), (id='bar', baz=(1, 2, 3));	
\end{lstlisting}
\vspace{40pt}

% --------------------------- LOAD
\subsection{LOAD}
\subsubsection{Description}
Takes result of $SELECT ... SAVE$ $outfile$ and inserts all the records into table.

\subsubsection{Synopsis}
\lstset{language=Java}
\begin{lstlisting}
LOAD filename INTO tablename
    [ THROTTLE throughput ]	
\end{lstlisting}

\begin{itemize}
	\item \textbf{filename} \\
	file containing records to upload
	\item \textbf{tablename} \\
	name of table
	\item \textbf{THROTTLE} \\
	limit the amount of throughput this query can consume\\
	$(read\_throughput, write\_throughput)$\\
	i.e. $(20\%, 50\%)$ or $(20, 50)$
\end{itemize}

\subsubsection{Examples}
\lstset{language=Java}
\begin{lstlisting}
LOAD archive.p INTO mytable;
LOAD dump.json.gz INTO mytable;	
\end{lstlisting}
\vspace{40pt}

% --------------------------- SCAN
\subsection{SCAN}
\subsubsection{Description}
Is the exact same as a SELECT statement with the difference that it's allowed to perform table scans.\\
See Section \ref{sssec:select}


% --------------------------- SELECT
\subsection{SELECT}
\label{sssec:select}
\subsubsection{Description}
Query table for items.

\subsubsection{Synopsis}
\lstset{language=Java}
\begin{lstlisting}
SELECT
    [ CONSISTENT ]
    attributes
    FROM tablename
    [ KEYS IN primary_keys | WHERE expression ]
    [ USING index ]
    [ LIMIT limit ]
    [ SCAN LIMIT scan_limit ]
    [ ORDER BY field ]
    [ ASC | DESC ]
    [ THROTTLE throughput ]
    [ SAVE filename]	
\end{lstlisting}

\begin{itemize}
	\item \textbf{CONSISTENT} \\
	if present, perform strongly consistent read
	\item \textbf{attributes} \\
	Comma-separated list of attributes to fetch or expressions\\
	$TIMESTAMP$ and $DATE$ functions can be used, as well as arithmetic ($foo + (bar - 3) / 100$)\\
	$SELECT *$ means ‘all attributes’\\
	$SELECT count(*)$ will return the number of results
	\item \textbf{tablename} \\
	name of table
	\item \textbf{index} \\
	When WHERE expression uses an indexed attribute, this allows you to manually specify which index name to use for the query
	\item \textbf{limit} \\
	maximum number of items to return
	\item \textbf{scan\_limit} \\
	maximum number of items for DynamoDB to scan
	\item \textbf{ORDER BY} \\
	sorts results by field
	\item \textbf{ASC | DESC} \\
	sorts results in ASCending (default) or DESCending order
	\item \textbf{THROTTLE} \\
	limit the amount of throughput this query can consume\\
	$(read\_throughput, write\_throughput)$\\
	i.e. $(20\%, 50\%)$ or $(20, 50)$
	\item \textbf{SAVE} \\
	saves results to a file
\end{itemize}

\subsubsection{Examples}
\lstset{language=Java}
\begin{lstlisting}
SELECT * FROM foobars SAVE out.p;
SELECT * FROM foobars WHERE foo = 'bar';
SELECT count(*) FROM foobars WHERE foo = 'bar';
SELECT id, TIMESTAMP(updated) FROM foobars KEYS IN 'id1', 'id2';
SELECT * FROM foobars KEYS IN ('hkey', 'rkey1'), ('hkey', 'rkey2');
SELECT CONSISTENT * foobars WHERE foo = 'bar' AND baz >= 3;
SELECT * foobars WHERE foo = 'bar' AND attribute_exists(baz);
SELECT * foobars WHERE foo = 1 AND NOT (attribute_exists(bar) OR contains(baz, 'qux'));
SELECT 10 * (foo - bar) FROM foobars WHERE id = 'a' AND ts < 100 USING ts-index;
SELECT * FROM foobars WHERE foo = 'bar' LIMIT 50 DESC;
SELECT * FROM foobars THROTTLE (50%, *);	
\end{lstlisting}
\vspace{20pt}

\subsubsection{WHERE Clause}
if provided, the SELECT operation will use these constraints as the KeyConditionExpression if possible, and if not (or if there are constraints left over), the FilterExpression. 
\vspace{40pt}

% --------------------------- UPDATE
\subsection{UPDATE}
\subsubsection{Description}
Updates items in table.

\subsubsection{Synopsis}
\lstset{language=Java}
\begin{lstlisting}
UPDATE tablename
    update_expression
    [ KEYS IN primary_keys ]
    [ WHERE expression ]
    [ USING index ]
    [ RETURNS (NONE | ( ALL | UPDATED) (NEW | OLD)) ]
    [ THROTTLE throughput ]	
\end{lstlisting}

\begin{itemize}
	\item \textbf{tablename} \\
	name of table
	\item \textbf{RETURNS} \\
	return items that were operated on
	\item \textbf{THROTTLE} \\
	limit the amount of throughput this query can consume\\
	$(read\_throughput, write\_throughput)$\\
	i.e. $(20\%, 50\%)$ or $(20, 50)$
	\item \textbf{SAVE} \\
	saves results to a file
\end{itemize}
\subsubsection{Examples}
\lstset{language=Java}
\begin{lstlisting}
UPDATE foobars SET foo = 'a';
UPDATE foobars SET foo = 'a', bar = bar + 4 WHERE id = 1 AND foo = 'b';
UPDATE foobars SET foo = if_not_exists(foo, 'a') RETURNS ALL NEW;
UPDATE foobars SET foo = list_append(foo, 'a') WHERE size(foo) < 3;
UPDATE foobars ADD foo 1, bar 4;
UPDATE foobars ADD fooset (1, 2);
UPDATE foobars REMOVE old_attribute;
UPDATE foobars DELETE fooset (1, 2);	
\end{lstlisting}
\vspace{20pt}